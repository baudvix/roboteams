\chapter{Fazit}\label{Fazit}

\section{Aufgabenstellung}
Die gegebene Aufgabenstellung umfasst viele Teilgebiete des Bachelor-Studiengangs Informatik an der Humboldt-Universität. Es werden sowohl gute Kenntnisse im Bereich Software-Engineering, Kommunikationssysteme und ein gutes Gefühl für Algorithmen benötigt als auch ein grundlegendes Wissen im Bereich mathematischer Problemstellungen, dass über die vermittelten Inhalte des Abiturs hinausgeht. Außerdem bietet sich die Möglichkeit einen Einblick in die Arbeit mit verschiedenen Robotern zu bekommen - wobei sich dieser Aspekt, auf Grund des kleinen Zeitfensters, nur auf sehr rudimentäre Anwendungen beschränkt.
Die Problemstellung ermöglicht eine gute und sinnvolle Unterteilung in Kleingruppen, die unter Einhaltung vereinbarter Schnittstellen unabhängig voneinander arbeiten können. Diese Teilaufgaben bieten weiterhin  die Möglichkeit, sich für ein gewisses Themengebiet zu entscheiden: Für die NXT's ist eine gewisse Kreativität und handwerkliches Geschick erforderlich, da die Modelle, bzw. das endgültige Modell, viele Anforderungen an seine Umgebung und seine Funktionalität erfüllen muss; Der NAO-Roboter bietet die Möglichkeit der Bilderkennung; die MCC-Gruppe hat hauptsächlich algorithmische Probleme zu lösen.
Wie bereits in der Aufgabenstellung zitiert, ist das Problem sehr Anwendungsorientiert - auch wenn es keinen direkten Forschungsbezug gibt, so gewinnt man doch einen sehr guten Eindruck davon, wie zukünftig Roboter bei realen Problemen aushelfen können und vor welchen Schwierigkeiten die Entwickler und Ingenieure stehen.

\section{Projektplanung}
Die größte Hürde der meisten Projekte ist die Projektplanung. Nicht nur ein Vorgehensmodell für die Softwareentwicklung, sowie eine exakte Erfassung von Anforderung sind hier notwendig, sondern auch eine gute Planung auf Ebene der gesamten Gruppe. Das heißt, eine unabhängige Kleingruppenarbeit ist nur möglich wenn die Schnittstellen einvernehmlich und präzise definiert wurden. Ein Integrations- und Systemtest ist nur möglich wenn alle Teilaufgaben bis zu einem vereinbarten Zeitpunkt fertig sind. Ein erfolgreicher Abschluss ist nur möglich wenn von Anfang ein genauer Projektplan ausgearbeitet wird, der es ermöglicht, bei nicht haltbaren Fristen entsprechend zu reagieren.
Auch in diesem Projekt hat sich die fehlende Erfahrung deutlich bemerkbar gemacht, insbesondere dadurch, dass letztendlich kein vollständig autonomer Ablauf bei der Abschlusspräsentation möglich war. Es ist offensichtlich, dass nicht nur Wissen in den oben genannten Fachgebieten von Nöten ist, sondern auch eine zumindest grundsätzliche Einführung in Projektmanagement wichtig wäre.
Die Projektplanung umfasst häufig auch den Einsatz bestimmter Werkzeuge während des Projekts. Hier bestand das Problem, dass es, abgesehen vom Versionierungstool ``GIT'', keine einheitliche Verwendung von Tools zur Dokumentation, Verteilung von Aufgaben, Software-Entwicklung und Software-Entwurf gab. 
Auch die Verwendung der verschiedenen Plattformen (MAC, Windows, Linux) bereitete insbesondere im Bereich der Visualisierung erhebliche Probleme.
Zuletzt scheint auch Entscheidung für Python als durchgängig zu verwende Programmiersprache keine gute gewesen zu sein. Allerdings muss an dieser Stelle gesagt werden, dass die damit verbunden Probleme keineswegs vorhersehbar gewesen sind.
\\
Abschließend lässt sich sagen, dass doch ein sehr erheblicher Teil der Arbeitszeit und des letztendlichen Projektergebnisses der ungenügenden Projektplanung zuzuschreiben sind - es wäre sicherlich sehr zielführend, wenn die Studierenden in zukünftigen Projekten explizit auf diese Herausforderung vorbereitet werden.

\section{Aussicht}
Aus unserer Sicht war das Semesterprojekt trotz der genannten Probleme nicht nur sehr lehrreich, sondern die meiste Zeit auch eine Bereicherung für die Teilnehmer. Insbesondere kann man aus den größeren aufgetretenen Problemen viel lernen, um es in zukünftigen Projekten besser zu machen.
Das Themengebiet bietet im allgemeinen unzählige Möglichkeiten an weiteren Projekten teilzunehmen, Abschlussarbeiten zu schreiben oder Forschung zu betreiben. Durch das Semesterprojekt hat man einen weit gefächerten Einblick in die Thematik erhalten. Die Aufgabe hat gezeigt, wie weit der Schritt zu einem tatsächlich autonom agierenden Roboterteam in einem unwegsamen Gelände noch ist.