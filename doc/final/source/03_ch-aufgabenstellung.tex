\chapter{Aufgabenstellung}\label{Aufgabenstellung}

\section{Formulierung der Aufgabe}

Die grobe Aufgabenstellung, wie sie ursprünglich ausgeschrieben war lautet:
``Bei Erkundungsaktionen in für Menschen schwer zugänglichen Gebieten - etwa in Katastrophengebieten wie der Region Fukushima - können künftig Teams von autonomen Robotern eingesetzt werden, die miteinander kooperieren. Ein Flugroboter erkundet dabei das Gebiet aus der Luft und gibt den am Boden agierenden Rettungsrobotern Lage- und Geländeinformationen. Damit solch ein Szenario reibungslos funktioniert, müssen eine Reihe software-technischer Probleme gelöst werden, von der Modellierung der Anforderungen über die Implementierung der Kommunikation und Sensorverarbeitung bis hin zur Konstruktion geeigneter Sensorik, Motorik und Verhalten der mobilen Plattformen. In diesem Semesterprojekt soll ein automomes Roboterteam, bestehend aus einem Quadrokopter, humanoiden Nao-Robotern und mehreren NXT-Robotern, entworfen und realisiert werden, welches eine kooperative Aufgabe selbständig in schwer zugänglichem Territorium durchführt. Schwerpunkt liegt dabei auf der Modellierung und dem Entwurf der Software für das verteilte System, welche nachweisbar bestimmte Korrektheitseigenschaften erfüllen muss. Begleitend zum Projekt werden das Seminar \textit{Schwarmverhalten} und die Vorlesung \textit{Software-Verifikation} angeboten. Die Teilnehmerzahl ist begrenzt. Von den Teilnehmern werden Vorkenntnisse in Programmierung und verteilten Systemen vorausgesetzt. Die Themen sind eng mit den Forschungsarbeiten der LS Kognitive Robotik und SVT verbunden und können zu Abschlussarbeiten führen. Besonders wichtig ist die Zusammenarbeit im Team.'' \footnote{https://goya3.informatik.hu-berlin.de/goyacs/course/showCourseDetails.do?id=8363\&caller=overview}\\
\\
Im Verlauf der ersten Treffen der Gruppenmitglieder mit den Professoren wurde diese grobe Aufgabenstellung konkretisiert, indem verschiedene Möglichkeiten, das Problem zu lösen, diskutiert wurden. Einerseits wurde zunächst von einem Flugroboter abgesehen, da die Auswertung bewegter Bilddaten bereits ein sehr anspruchsvolles und umfangreiches Problem für sich allein ist - optional konnte dieser bei ausreichenden Ressourcen noch hinzu genommen werden. Das Gebiet indem der Zielbereich gefunden werden soll, kann durch Wände oder andere glatte Oberflächen begrenzt werden; die Hindernisse sollen durch Ziegel oder ähnliche, gleich beschaffene, Steine modelliert werden. Auch hier ist eine Anpassung des Schwierigkeitsgrades durch die Anzahl der Steine und die Größe des Spielfelds möglich. Eine zeitliche Beschränkung besteht automatisch durch die Akkulebensdauer der Roboter (s. dazu Abschnitt Nao, bzw. NXT). Die erste Idee für die allgemeine Vorgehensweise ist, die NXT's für die Erkundung der Umgebung zu verwenden, die dadurch erhaltenen Daten durch ein steuerndes System auszuwerten und anschließend zu benutzen, um den NAO zum Ziel zu führen. Die genauen Rahmenbedingungen durch die Aufgabe und die Einschränkungen und Anforderungen an das System und die Roboter sind in der Anforderungserhebung formuliert (siehe Kapitel \ref{Anforderungserhebung} auf Seite \pageref{Anforderungserhebung}).


\section{Verfügbare Ressourcen}

Für das Projekt sind laut Studienordnung 12 SP angesetzt - dies entspricht etwa einem Aufwand von 360 Stunden, also etwa 25 bis 30 Stunden pro Person pro Woche. Zusätzlich haben Frau Professor Hafner, Herr Professor Schlingloff und Marcus Scheuenemann einmal wöchentlich an einer gemeinsamen Sitzung von 90 Minuten teilgenommen und waren darüber hinaus auch während des gesamten Semesters als Ansprechpartner verfügbar.
Die für das Projekt eingeplanten Roboter waren zwei NAO's, sowie mehrere NXT-Bausätze, die je nach Bauart ausreichend für 3-5 Roboter sind.
Für Test- und Arbeitszwecke wurde ein Raum im Fraunhofer Institut in Berlin-Adlershof zur Verfügung gestellt, der uneingeschränkt genutzt werden konnte.
Weiterhin wurden zur Ausführung und Entwicklung der Software die privaten Rechner der Studenten verwendet - Lizenzen für eine Python-Entwicklungsumgebung wurden von der Universität zur Verfügung gestellt.
