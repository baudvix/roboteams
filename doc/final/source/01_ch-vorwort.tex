\chapter{Vorwort}\label{Vorwort}

Bei der vorliegende Projektdokumentation handelt es sich um die abschließende Dokumentation des Semesterprojektes ``Softwaretechnik für autonome Roboterteams'' im Sommersemester 2012 am Institut für Informatik der Humboldt-Universität zu Berlin. Das Projekt ist verpflichtender Bestandteil zum Erlangen des Bachelor of Science am Institut.\\
Diese Dokumentation wurde Frau Prof. Dr. sc. nat. Verena Hafner und Herrn Prof. Dr. rer. nat. habil. Bernd-Holger Schlingloff, den Betreuern des Projekts, vorgelegt. Durchgeführt wurde das Projekt von Jonathan Sielhorst, Jens Bork, Sven Schröder, Lorenz Fichte, Denis Erfurt, Cordt Voigt, Robert Fritz, Josef Hufnagl und Sebastian Günther. Außerdem wurde das Projekt von Marcus Scheunemann vom Lehrstuhl Kognitive Robotik betreut.
In der Studienordnung \footnote{http://www.informatik.hu-berlin.de/institut/dokumente/ordnungen} ist das Semesterprojekt folgendermaßen beschrieben: ``[...] Studierende üben die Fähigkeit, in einem Team ein komplexes System, das eine gegebene Aufgabenstellung löst, in Hard- und/oder Software zu entwerfen, zu entwickeln, zu testen und zu dokumentieren sowie die Ergebnisse in geeigneter Form zu präsentieren. Sie erlangen Kenntnisse über die typischen Probleme bei Projekten mit mehr als 2 Beteiligten. Sie erhalten die Fähigkeit zur selbstkritischen Präsentation des Erreichten und der vorgenommenen Entscheidungen.''\\
Diese Dokumentation soll also nicht nur den Zweck haben das Problem und die Lösungstrategie zu beschreiben und zu  dokumentieren, sondern auch die während des Projekts aufgetretenen Probleme, Heransgehensweisen und Schwierigkeiten beleuchten und somit anderen Studierenden die Möglichkeit geben daraus zu profitieren.
Berlin, Sevilla, Zürich, den 10.10.2012