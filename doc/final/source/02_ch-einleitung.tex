\chapter{Einleitung}

Das in den nachfolgenden Kapiteln beschriebene Projekt wurde in einer vollständig heterogenen Soft- und Hardwareumgebung durchgeführt, es wird daher weitestgehend auf betriebssystem-, software- oder hardwarespezifische Beschreibungen verzichtet - da es jedoch auch Teilaufgaben gab, bei denen ein homogenes Umfeld maßgeblich für einen korrekten Ablauf war, werden diese Notwendigkeiten gesondert erläutert. Insbesondere die Arbeitsumgebungen der in diesem Projekt verwendeten Roboter werden genauer erläutert.\\
Die Programmiersprache die durchgehend verwendet wird ist Python in der Version 2.7; lediglich bei der hardwarenahen Ansteuerung einer Robotergruppe wurde aus pragmatischen Gründen NXC (Not excatly C,\footnote{http://bricxcc.sourceforge.net/nbc/})verwendet.
Verwendete anwendungsspezifische Begriffe werden gesondert im Glossar erläutert, ein grundsätzliches Verständnis der Themengebiete Informatik und Programmierung wird allerdings vorrausgesetzt.
