\section{Was ist LEGO(TM) Mindstroms(TM)}
\section{der Auftrag für die NXTs}
\section{die Konstruktion -- von der ersten Idee zum funktionstüchtigen Erkundungsroboter}
\subsection{Software}
\subsubsection{NXC}
\subsubsection{nxt-python}
\subsubsection{hybrider Ansatz}
\subsection{Idee 1 $\rightarrow$ Modell 1}%Kettenfahrzeug + Radar
\subsubsection{Idee}
Unsere erste Idee bestand im Prinzip aus zwei unabhängigen Ideen. Zum Einen wollten wir ein Fahrgestell konzipieren, das auch bei unwegsamen Gelände eine kontrollierte Bewegung des Explorer ermöglichen würde und zum Anderen wollten wir einen Sensor der schon viele Informationen über die Umgebung sammelt ohne, dass der Explorer jeden Quadratzentimeter abfahren muss.
\subsubsection{Konstruktion}
\subsubsection{Test}
\subsubsection{Pros \& Cons}
\subsubsection{Fazit}
\subsection{Idee 2 $\rightarrow$ Modell 2}%angepasstes Grundmodell + riesiger Stoßdämpfer
\subsubsection{Idee}
\subsubsection{Konstruktion}
\subsubsection{Test}
\subsubsection{Pros \& Cons}
\subsubsection{Fazit}
\subsection{Idee 3 $\rightarrow$ Modell 3}%angepasstes Grundmodell + verbesserte Stoßdämpfer
\subsubsection{Idee}
\subsubsection{Konstruktion}
\subsubsection{Test}
\subsubsection{Pros \& Cons}
\subsubsection{Fazit}
\subsection{Fazit und Entscheidung}
\section{Kommunikation}
\subsection{Bluetooth}
\subsection{Kommunikationsprotokoll PC $\leftrightarrow$ NXT}
\subsection{Kommunikation mit dem MCC}
