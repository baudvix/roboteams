\documentclass[10pt,a4paper]{scrartcl}
\usepackage[utf8x]{inputenc}
\usepackage{ngerman}
\usepackage{amsmath}
\usepackage{amsfonts}
\usepackage{amssymb}
\usepackage{makeidx}
\usepackage{graphicx}
\usepackage[left=2cm,right=2cm,top=2cm,bottom=2cm]{geometry}
\author{Sven Schröder}
\title{Dokumentation -- NXT}
\begin{document}
\maketitle
\tableofcontents
\section{Entwurf}
\subsection{Software}
Beim Entwurf der Software für unseren Teil der Aufgabe hatten wir zwei Dinge zu beachten, die mäßige Rechenleistung des LEGO$^\copyright$ Mindstorms$^\copyright$ NXT Brick (im folgenden nur noch Brick genannt) und die durch LEGO$^\copyright$ begrenzte Anzahl von Robotern auf maximal 4.  
\subsubsection{nxc -- Not eXactly C}
\subsubsection{nxt-python-framework}
\subsubsection{hybrider Ansatz}
\subsection{Idee 1 $\rightarrow$ Modell 1}%Kettenfahrzeug + Radar
\subsubsection{Idee}
Unsere erste Idee bestand im Prinzip aus zwei unabhängigen Ideen. Zum Einen wollten wir ein Fahrgestell konzipieren, das auch bei unwegsamen Gelände eine kontrollierte Bewegung des Explorer ermöglichen würde und zum Anderen wollten wir einen Sensor der schon viele Informationen über die Umgebung sammelt ohne, dass der Explorer jeden Quadratzentimeter abfahren muss.
\subsubsection{Konstruktion}
\subsubsection{Test}
\subsubsection{Pros \& Cons}
\subsubsection{Fazit}
\subsection{Idee 2 $\rightarrow$ Modell 2}%angepasstes Grundmodell + riesiger Stoßdämpfer
\subsubsection{Idee}
\subsubsection{Konstruktion}
\subsubsection{Test}
\subsubsection{Pros \& Cons}
\subsubsection{Fazit}
\subsection{Idee 3 $\rightarrow$ Modell 3}%angepasstes Grundmodell + verbesserte Stoßdämpfer
\subsubsection{Idee}
\subsubsection{Konstruktion}
\subsubsection{Test}
\subsubsection{Pros \& Cons}
\subsubsection{Fazit}
\subsection{Fazit und Entscheidung}
\section{Kommunikation}
\subsection{Bluetooth}
\subsection{Kommunikationsprotokoll PC $\leftrightarrow$ NXT}
\subsection{Kommunikation mit dem MCC}
\section{Logik}
\subsection{Explorationsalgorithmen}
\subsubsection{Exploration -- simple}
\subsubsection{Exploration -- circle}
\subsubsection{Exploration -- radar}
\subsection{GoToPoint}
\end{document}
